\documentclass{article}

\usepackage{graphicx}
\usepackage{indentfirst}
\usepackage{booktabs}
\usepackage[a4paper, total={6in, 8in}]{geometry}
\usepackage{hyperref}
\usepackage{fancyhdr}
\usepackage{subcaption}
\usepackage{xepersian}
\usepackage{fontspec}
\usepackage{url}
\settextfont[Scale=1.2,ExternalLocation=fonts/,BoldFont=B Nazanin Bold.ttf]{B Nazanin}
\setlatintextfont[Scale=1.2,ExternalLocation=fonts/,BoldFont=XB Zar.ttf]{Times New Roman}

\begin{document}


%title page%
\begin{titlepage}
	\begin{center}
		\vspace{0.2cm}
		
		\includegraphics[width=0.4\textwidth]{sharif.png}\\
		\vspace{0.5cm}
		\textbf{ \Huge{فاز دوم}}\\
		\vspace{0.25cm}
		\textbf{ \Large{پروژه مقدمه‌ای بر بیوانفورماتیک \\ دکتر علی  شریفی‌زارچی و دکتر سمیه کوهی}}
		\vspace{0.2cm}
		
		
		\large \textbf{دانشکده مهندسی کامپیوتر}\\\vspace{0.1cm}
		\large   دانشگاه صنعتی شریف\\\vspace{0.2cm}
		\large   ﻧﯿﻢ‌سال اول ۰۱-۰۲ \\\vspace{0.2cm}
		\large{\Large{امیرحسین باقری - 98105621}}\\
		\large{\Large{مهدی مستانی - 97100513}}\\
		\large{\Large{محمدرضا مفیضی - 98106059}}\\
	\end{center}
\end{titlepage}
%title page%

\newpage
\tableofcontents
\newpage
%pages header
\pagestyle{fancy}
\fancyhf{}
\fancyfoot{}
\setlength{\headheight}{59pt}
\cfoot{\thepage}
\lhead{فاز دوم}
\rhead{\includegraphics[width=0.1\textwidth]{sharif.png}\\
		دانشکده مهندسی کامپیوتر
}
\chead{پروژه مقدمه‌ای بر بیوانفورماتیک}
%pages header

\section{بیان ژن‌ها}
\subsection{انتخاب تمام نمونه‌های سالم}
در جدول \ref{tab:genes} ژن‌هایی که بیان آن‌ها به طرز معناداری بین گروه سالم و گروه بیماران متفاوت است را نمایش می‌دهیم
\footnote{جدول کامل به همراه این گزارش ضمیمه شده است.}.

\begin{table}[h!]
	\begin{latin}
		\begin{center}
			\begin{tabular}{@{}cccc@{}}
				\toprule
				Gene Symbol & Gene ID & Adj. P-Val & logFC  \\ \midrule
				MPO         & 4353    & 3.617e-19 & 5.563 \\
				FLT3        & 2322    & 4.835e-19 & 5.250 \\
				KIAA0101    & 9768    & 6.308e-19 & 4.559 \\
				BUB1B       & 701     & 1.664e-18 & 2.756 \\
				SUCNR1      & 56670   & 1.938e-18 & 2.996 \\ \bottomrule
			\end{tabular}
		\end{center}
	\end{latin}
	\caption{5 ژنی که بیان آن‌ها به طرز معناداری بین گروه سالم و گروه بیماران متفاوت است.}
	\label{tab:genes}
\end{table}

حالا ژن‌هایی که در بیماران از بقیه بیشتر بیان شده‌اند و همچنین ژن‌هایی که کمتر از بقیه بیان شده‌اند را در جدول \ref{tab:up-down} نمایش می‌دهیم
\footnote{جدول کامل این بخش هم به همراه گزارش ضمیمه شده است.}.

\begin{table}[h!]
	\begin{latin}
		\begin{center}
			\begin{tabular}{@{}cc@{}}
				\toprule
				Least Expressed & Most Expressed \\ \midrule
				STK38          & MPO             \\
				CBX7           & FLT3            \\
				PLCL2          & KIAA0101        \\
				PECR           & BUB1B           \\
				HLA-F          & SUCNR1          \\ \bottomrule
			\end{tabular}
		\end{center}
	\end{latin}
	\caption{5 ژنی که در بیماران بیشتر/کمتر از بقیه بیان شده‌اند.}
	\label{tab:up-down}
\end{table}

\subsection{انتخاب نمونه‌های سالم با بیشترین هم‌بستگی}
در جدول \ref{tab:genes-2} ژن‌هایی که بیان آن‌ها به طرز معناداری بین گروه سالم و گروه بیماران متفاوت است را نمایش می‌دهیم.

\begin{table}[h!]
	\begin{latin}
		\begin{center}
			\begin{tabular}{@{}cccc@{}}
				\toprule
				Gene Symbol & Gene ID & Adj. P-Val & logFC  \\ \midrule
				MPO      & 4353  & 7.714e-16 & 5.73  \\
				PECR     & 55825 & 7.714e-16 & -2.314  \\
				KIT      & 3815  & 7.126e-15 & 5.262 \\
				KIAA0101 & 9768  & 7.969e-15 & 4.308 \\
				FLT3     & 2322  & 7.969e-15 & 5.155 \\ \bottomrule
			\end{tabular}
		\end{center}
	\end{latin}
	\caption{5 ژنی که بیان آن‌ها به طرز معناداری بین گروه سالم و گروه بیماران متفاوت است.}
	\label{tab:genes-2}
\end{table}

حالا ژن‌هایی که در بیماران از بقیه بیشتر بیان شده‌اند و همچنین ژن‌هایی که کمتر از بقیه بیان شده‌اند را در جدول \ref{tab:up-down-2} نمایش می‌دهیم.

\begin{table}[h!]
	\begin{latin}
		\begin{center}
			\begin{tabular}{@{}cc@{}}
				\toprule
				Least Expressed & Most Expressed \\ \midrule
				PECR           & MPO             \\
				DDX58          & KIT            \\
				CBX7           & KIAA0101        \\
				STK38          & FLT3           \\
				NR1D2          & SUCNR1          \\ \bottomrule
			\end{tabular}
		\end{center}
	\end{latin}
	\caption{5 ژنی که در بیماران بیشتر/کمتر از بقیه بیان شده‌اند.}
	\label{tab:up-down-2}
\end{table}

\section{بررسی \lr{pathway} و \lr{gene ontology}}
\subsection{ژن‌هایی که کمتر بیان شده‌اند}
با رفتن به سایت \href{https://maayanlab.cloud/Enrichr/}{\lr{enrichr}} و قرار دادن لیست ژن‌های کمتر بیان‌شده (شکل \ref{fig:enrichr}) آن‌‌ها را بررسی می‌کنیم.
\begin{figure}[h!]
	\centering
	\includegraphics[width=0.5\columnwidth]{figs/enrichr.jpg}
	\caption{وارد کردن داده‌ها برای شروع آنالیز}
	\label{fig:enrichr}
\end{figure}

\subsubsection{انتخاب تمام نمونه‌های سالم}
در بخش \lr{pathways} (شکل \ref{fig:enrichr-pathways}) می‌توان پایگاه‌داده‌های \lr{pathway} مختلف را بررسی کرد.
\begin{figure}[h!]
	\centering
	\includegraphics[width=0.5\columnwidth]{figs/enrichr-pathways.jpg}
	\caption{بخش \lr{pathways} در \lr{enrichr} برای ژن‌های کمتر بیان‌شده و تمام نمونه‌ها}
	\label{fig:enrichr-pathways}
\end{figure}

به‌عنوان مثال در پایگاه‌داده \lr{Reactome 2022} (شکل \ref{fig:enrichr-pathways-reactome}) تعداد زیادی از ژن‌هایی که میزان بیان کمتری داشتند مربوط به \lr{pathway} سیستم ایمنی \LTRfootnote{\lr{Immune System}} هستند.
\begin{figure}[h!]
	\begin{subfigure}{.59\columnwidth}
		\centering
		\includegraphics[width=\columnwidth]{figs/enrichr-pathways-reactome.jpg}
	\end{subfigure}
	\begin{subfigure}{.41\columnwidth}
		\centering
		\includegraphics[width=\columnwidth]{figs/enrichr-pathways-reactome-tab.jpg}
	\end{subfigure}
	\caption{\lr{pathway}های پایگاه‌داده \lr{Reactome}}
	\label{fig:enrichr-pathways-reactome}
\end{figure}

و همچنین پایگاه‌داده \lr{KEGG 2021} در شکل \ref{fig:enrichr-pathways-kegg} نمایش داده شده است.
\begin{figure}[h!]
	\begin{subfigure}{.53\columnwidth}
		\centering
		\includegraphics[width=\columnwidth]{figs/enrichr-pathways-kegg.jpg}
	\end{subfigure}
	\begin{subfigure}{.47\columnwidth}
		\centering
		\includegraphics[width=\columnwidth]{figs/enrichr-pathways-kegg-tab.jpg}
	\end{subfigure}
	\caption{\lr{pathway}های پایگاه‌داده \lr{KEGG}}
	\label{fig:enrichr-pathways-kegg}
\end{figure}

با مراجعه به سایت \href{https://www.genome.jp}{\lr{KEGG}} و جست‌و‌جوی معنادارترین \lr{pathway} یعنی
\lr{NF-kappa B signaling pathway}
تصویر آن را در شکل \ref{fig:kegg-pathway} نمایش می‌دهیم.
\begin{figure}[h!]
	\centering
	\includegraphics[width=0.5\columnwidth]{figs/kegg-pathway.jpg}
	\caption{شمای \lr{NF-kappa B signaling pathway}}
	\label{fig:kegg-pathway}
\end{figure}

و با مراجعه به سایت \href{https://reactome.org}{\lr{reactome}} و جست‌و‌جوی معنادارترین \lr{pathway} یعنی
\lr{Immune System R-HSA-168256}
تصویر آن را در شکل \ref{fig:reactome-pathway} نمایش می‌دهیم.
\begin{figure}[h!]
	\centering
	\includegraphics[width=0.6\columnwidth]{figs/reactome-pathway.pdf}
	\caption{شمای \lr{Immune System R-HSA-168256}}
	\label{fig:reactome-pathway}
\end{figure}

در بخش \lr{ontologies} می‌توان \lr{biological processes} (شکل \ref{fig:enrichr-ontology-bp}) و \lr{cellular components} (شکل \ref{fig:enrichr-ontology-cc}) و \lr{molecular functions} (شکل \ref{fig:enrichr-ontology-mf}) را بررسی کرد (شکل \ref{fig:enrichr-ontology}).
\begin{figure}[h!]
	\centering
	\includegraphics[width=0.5\columnwidth]{figs/enrichr-ontologies.jpg}
	\caption{بخش \lr{ontologies} در \lr{enrichr} برای ژن‌های کمتر بیان‌شده و تمام نمونه‌ها}
	\label{fig:enrichr-ontology}
\end{figure}

\begin{figure}[h!]
	\centering
	\includegraphics[width=0.5\columnwidth]{figs/enrichr-ontologies-bp.jpg}
	\caption{\lr{biological processes} در بخش \lr{ontologies}}
	\label{fig:enrichr-ontology-bp}
\end{figure}

\begin{figure}[h!]
	\centering
	\includegraphics[width=0.5\columnwidth]{figs/enrichr-ontologies-cc.jpg}
	\caption{\lr{cellular components} در بخش \lr{ontologies}}
	\label{fig:enrichr-ontology-cc}
\end{figure}

\begin{figure}[h!]
	\centering
	\includegraphics[width=0.5\columnwidth]{figs/enrichr-ontologies-mf.jpg}
	\caption{\lr{molecular functions} در بخش \lr{ontologies}}
	\label{fig:enrichr-ontology-mf}
\end{figure}

\subsubsection{انتخاب نمونه‌های سالم با بیشترین هم‌بستگی}
دوباره با رفتن به بخش \lr{pathways} (شکل \ref{fig:enrichr-pathways-2}) می‌توان پایگاه‌داده‌های \lr{pathway} مختلف را این‌بار برای نمونه‌های جدید بررسی کرد.
\begin{figure}[h!]
	\centering
	\includegraphics[width=0.5\columnwidth]{figs/enrichr-pathways-2.jpg}
	\caption{بخش \lr{pathways} در \lr{enrichr} برای ژن‌های کمتر بیان‌شده و نمونه‌های با هم‌بستگی بیشتر}
	\label{fig:enrichr-pathways-2}
\end{figure}

پایگاه‌داده \lr{Reactome 2022} و همچنین پایگاه‌داده \lr{KEGG 2021} در شکل \ref{fig:enrichr-pathways-reactome-kegg} نمایش داده شده است.

\begin{figure}[h!]
	\begin{subfigure}{.5\columnwidth}
		\centering
		\includegraphics[width=\columnwidth]{figs/enrichr-pathways-reactome-2.jpg}
	\end{subfigure}
	\begin{subfigure}{.5\columnwidth}
		\centering
		\includegraphics[width=\columnwidth]{figs/enrichr-pathways-kegg-2.jpg}
	\end{subfigure}
	\caption{\lr{pathway}های پایگاه‌داده \lr{Reactome} و \lr{KEGG}}
	\label{fig:enrichr-pathways-reactome-kegg}
\end{figure}

در بخش \lr{ontologies} می‌توان \lr{biological processes} و \lr{cellular components} و \lr{molecular functions} را بررسی کرد (شکل \ref{fig:enrichr-ontology-2}).
\begin{figure}[h!]
	\centering
	\includegraphics[width=0.5\columnwidth]{figs/enrichr-ontologies-2.jpg}
	\caption{بخش \lr{ontologies} در \lr{enrichr} برای ژن‌های کمتر بیان‌شده و نمونه‌های با هم‌بستگی بیشتر}
	\label{fig:enrichr-ontology-2}
\end{figure}

\subsection{ژن‌هایی که بیشتر بیان شده‌اند}
حالا لیست ژن‌های بیشتر بیان‌شده را بررسی می‌کنیم.

\subsubsection{انتخاب تمام نمونه‌های سالم}
در بخش \lr{pathways} (شکل \ref{fig:enrichr-pathways-d}) می‌توان پایگاه‌داده‌های \lr{pathway} مختلف را بررسی کرد.
\begin{figure}[h!]
	\centering
	\includegraphics[width=0.5\columnwidth]{figs/enrichr-pathways-d.jpg}
	\caption{بخش \lr{pathways} در \lr{enrichr} برای ژن‌های بیشتر بیان‌شده و تمام نمونه‌ها}
	\label{fig:enrichr-pathways-d}
\end{figure}

در بخش \lr{ontologies} می‌توان \lr{biological processes} و \lr{cellular components} و \lr{molecular functions} را بررسی کرد (شکل \ref{fig:enrichr-ontology-d}).
\begin{figure}[h!]
	\centering
	\includegraphics[width=0.5\columnwidth]{figs/enrichr-ontologies-d-2.jpg}
	\caption{بخش \lr{ontologies} در \lr{enrichr} برای ژن‌های بیشتر بیان‌شده و تمام نمونه‌ها}
	\label{fig:enrichr-ontology-d}
\end{figure}

\subsubsection{انتخاب نمونه‌های سالم با بیشترین هم‌بستگی}
دوباره با رفتن به بخش \lr{pathways} (شکل \ref{fig:enrichr-pathways-d-2}) می‌توان پایگاه‌داده‌های \lr{pathway} مختلف را این‌بار برای نمونه‌های جدید بررسی کرد.
\begin{figure}[h!]
	\centering
	\includegraphics[width=0.5\columnwidth]{figs/enrichr-pathways-d-2.jpg}
	\caption{بخش \lr{pathways} در \lr{enrichr} برای ژن‌های بیشتر بیان‌شده و نمونه‌های با هم‌بستگی بیشتر}
	\label{fig:enrichr-pathways-d-2}
\end{figure}

در بخش \lr{ontologies} می‌توان \lr{biological processes} و \lr{cellular components} و \lr{molecular functions} را بررسی کرد (شکل \ref{fig:enrichr-ontology-d-2}).
\begin{figure}[h!]
	\centering
	\includegraphics[width=0.5\columnwidth]{figs/enrichr-ontologies-d-2.jpg}
	\caption{بخش \lr{ontologies} در \lr{enrichr} برای ژن‌های بیشتر بیان‌شده و نمونه‌های با هم‌بستگی بیشتر}
	\label{fig:enrichr-ontology-d-2}
\end{figure}

\section{جست‌و‌جو در مقالات}
\subsection*{الف)}
در مقاله \cite{e2f4} و بخش نتیجه‌گیری بیان شده است که ژن \lr{E2F4} (که در پایگاه‌داده \lr{TRANSFAC and JASPAR PWMs} به‌عنوان معنادارترین ژن با بیان زیاد انتخاب شده است) بیش از حد بیان شده است
\LTRfootnote{\lr{E2F4 is aberrantly overexpressed in AML and is associated with poor prognosis}}
. 

در مقاله‌های \cite{irf8, irf8-2} گفته شده که ژن \lr{IRF8} (که در پایگاه‌داده \lr{ENCODE  and ChEA Consensus TFs from} آمده است) کمتر بیان شده است
\LTRfootnote{\lr{We validated the critical role of the transcription factor IRF8 and demonstrated that it modulates the function of the cells by regulating important signaling molecules.}}
. 

در مقاله \cite{interferon} اشاره شده است که \lr{Interferon Alpha/Beta Signaling R-HSA-909733} در بیماران مبتلا به \lr{AML} به میزان قابل توجهی کم بیان شده است (مطابق شکل \ref{fig:enrichr-pathways-reactome}) 
\LTRfootnote{\lr{The absence of monocyte and pDC activation by IFNα ex vivo could explain the lack of an in vivo anti-leukemic effect, and the therapeutic effect of IFNα may potentially be enhanced by removing this inherent block of activation in healthy immune subsets in AML patients}}
.

در مقاله \cite{cell-cycle} نیز اشاره شده که \lr{Cell cycle} در کنترل بیماری موثر است (شکل \ref{fig:enrichr-pathways-d})
\LTRfootnote{\lr{Cell cycle control in acute myeloid leukemia}}
.

\subsection*{ب)}
در مقاله \cite{irf8} به دارویی اشاره شده است که برای درمان \lr{AML} باید مقدار بیان ژن \lr{IRF8} را افزایش داد
\LTRfootnote{\lr{As an inhibitor for the related protein IRF5 was recently described, IRF8 may also be a suitable drug target.}}
.

داروی \lr{Arsenic trioxide} که یکی از موثرترین داروهای در درمان بیماری \lr{AML} است در \cite{kappa} بیان شده است که \lr{NF-kappa B signaling pathway} از اهداف اصلی این دارو است (شکل \ref{fig:kegg-pathway} و \ref{fig:kappa}).

\begin{figure}[h!]
	\centering
	\includegraphics[width=0.5\columnwidth]{figs/kappa.jpg}
	\caption{مشاهده \lr{NF-kappa B signaling pathway} در هدف‌های موثر از داروی \lr{Arsenic trioxide}}
	\label{fig:kappa}
\end{figure}

\clearpage

\begin{thebibliography}{99}
	\begin{latin}		
		\bibitem{e2f4}
		Feng, Y, Li, L, Du, Y, Peng, X, Chen, F. (2020) \emph{E2F4 functions as a tumour suppressor in acute myeloid leukaemia via inhibition of the MAPK signalling pathway by binding to EZH2}, Cell Mol Med.
		
		\bibitem{interferon}
		Forthun, R.B., Hellesøy, M., Sulen, A. et al. (2019) \emph{Modulation of phospho-proteins by interferon-alpha and valproic acid in acute myeloid leukemia}, Cancer Res Clin Oncol.
		
		\bibitem{cell-cycle}
		Schnerch D, Yalcintepe J, Schmidts A, et al. (2012) \emph{Cell cycle control in acute myeloid leukemia}, Am J Cancer Res.
		
		\bibitem{irf8-2}
		Sharma A, Yun H, Jyotsana N, Chaturvedi A, Schwarzer A, Yung E, Lai CK, Kuchenbauer F, Argiropoulos B, Görlich K, Ganser A, Humphries RK, Heuser M. (2015) \emph{Constitutive IRF8 expression inhibits AML by activation of repressed immune response signaling. Leukemia}, Epub.
		
		\bibitem{irf8}
		Liss F, Frech M, Wang Y, et al. (2021) \emph{IRF8 Is an AML-Specific Susceptibility Factor That Regulates Signaling Pathways and Proliferation of AML Cells}, Cancers (Basel).
		
		\bibitem{kappa}
		Trisenox. \emph{Arsenic trioxide}, Drug Bank \url{https://go.drugbank.com/drugs/DB01169}.
		
		
		
%		\bibitem{lamport94
%		Leslie Lamport (1994) \emph{\LaTeX: a document preparation system}, Addison
%		Wesley, Massachusetts, 2nd ed.
	\end{latin}
\end{thebibliography}



\end{document}
